\documentclass{article} 
\usepackage[utf8]{inputenc}
\usepackage{graphicx}
\usepackage{enumitem}
\usepackage{mathtools, amssymb, amsthm} % imports amsmath

\author{Daniel Palma}
\date{\today}
\title{Test 3 Notes}

\begin{document}

\maketitle
\newpage

\tableofcontents
\newpage

\section{Change of Variables (Jacobians)}

In one-dimensional calculus we often use a change of variable to simplify an integrals.

A change of variables can also be useful in double integrals. We have already seen one example of this: conversion to polar coordinates. The new variables $r$ and $\theta$ are related to the old variables $x$ and $y$ by the equations 

$$ x = r \cos{\theta} \qquad y = r \sin{\theta} $$

and the change of variables formula can we written as 

$$\iint\limits_{R} f(x,y) \mathrm{d}A = \iint\limits_{S} f(r\cos{\theta}, r\sin{\theta})r \ dr \ d\theta$$

Where $S$ is the region in the $r\theta$-plane that corresponds to the region $R$ in the $xy$-plane

More generally, we consider a change of variables that is given by a \textbf{transformation} $T$ from the $uv$-plane to the $xy$-plane:

$$x = g(u,v) \qquad y = h(u,v)$$

or, as we sometimes write,

$$x = x(u,v) \qquad y = y(u,v)$$

We usually assume that $T$ is a \textbf{$\mathbf{C^1}$ transformation}, which means that $g$ and $h$ have continuous first-order partial derivatives.

A transformation $T$ is really just a function whose domain and range are both subsets of $\mathbb{R}^2$. If $T(u_1, v_1) = (x_1.y_1)$, then the point $(x_1,y_1)$ is called the \textbf{image} of the point $(u_1, v_1)$. If no two points have the same image, $T$ is called \textbf{one-to-one}. 

If $T$ is a one-to-one transformation, then it has an \textbf{inverse transformation $\mathbf{T^-1}$} from the $xy$-plane to the $uv$-plane and it may be possible to solve 

$$x = x(u,v) \qquad y = y(u,v)$$

for $u$ and $v$ in terms of $x$ and $y$:

$$u = G(x,y) \qquad v = H(x,y)$$

\subsection{Jacobian}

The \textbf{Jacobian} of the transformation $T$ is given by $x = g(u,v)$ and $y = h(u,v)$ is 


$$\frac{\partial(x,y)}{\partial(u,v)} = 
\begin{vmatrix}
    \frac{\partial x}{\partial u} & \frac{\partial x}{\partial v} \\[8pt]
    \frac{\partial y}{\partial u} & \frac{\partial y}{\partial v} 
\end{vmatrix} = \frac{\partial x}{\partial u} \frac{\partial y}{\partial v} - \frac{\partial x }{ \partial v} \frac{\partial y}{ \partial u}
$$


\subsection{Change of Variables in a Double Integrals}


Supopse that $T$ is a $C^1$ tranformation whose Jacobian is nonzero and that $T$ maps a region $S$ in the $uv$- plane onto a region $R$ in the $xy$-plane. Supposed that $f$ is continuous on $R$ and that $R$ and $S$ are type I or type II plane regions. Suppose also that $T$ is one-to-one, except perhaps on the boundary of $S$. Then 

$$\iint\limits_{R}f(x,y)\ \mathrm{d}A = \iint\limits_{S}f(x(u,v), y(u,v))\begin{vmatrix}
    \frac{\partial(x,y)}{\partial(u,v)}
\end{vmatrix} \ du \ dv$$

\subsubsection{Triple Integrals}

Lets use the definition of the Jacobian, extend it to three dimensions and find the formula for a triple integral and use it to derive the formula for spherical coordinates.

The Jacobian of $T$ is the following $3 \times 3 $ determinant:

$$\frac{\partial(x,y,z)}{\partial(u,v,w)} = \begin{vmatrix}
    \frac{\partial x}{\partial u} & \frac{\partial x}{\partial v} & \frac{\partial x}{ \partial w} \\[6pt]
    \frac{\partial y}{\partial u} & \frac{\partial y}{ \partial v} & \frac{\partial y}{ \partial w} \\[6pt]
    \frac{\partial z}{ \partial u} & \frac{\partial z }{\partial v} & \frac{\partial z}{ \partial w} 
\end{vmatrix}$$

this gives us the MASSIVE formula lol:

$$\iiint\limits_{R} f(x,y,z) \ \mathrm{d}V = \iiint\limits_{S} f(x(u,v,w), y(u,v,w), z(u,v,w)) \bigg\rvert \frac{\partial(x,y,z)}{\partial(u,v,w)} \bigg\rvert \ du \ dv \ dw$$

now lets use this to find the formula for triple intrgration in spherical coordinates!!!! (i'm losing my fucking mind) 

$$x =  \rho \sin{\phi} \cos{\theta} \qquad y = \rho \sin{\phi} \sin{\theta} \qquad z = \rho \cos{\phi}$$

lets compute this absolute unit of a jacobian 

\begin{align*}
    \frac{\partial(x,y,z)}{\partial(\rho,\theta,\phi)} &= \begin{vmatrix}
    \sin{\phi}\cos{\theta}  & -\rho\sin{\phi}\sin{\theta}  & \rho \cos{\phi} \cos{\theta} \\[6pt]
    \sin{\phi}\sin{\theta}   & \rho \sin{\phi} \cos{\theta} & \rho \cos{\phi}\sin{\theta}  \\[6pt]
    \cos{\phi}  & 0 & - \rho \sin(\phi)    
    \end{vmatrix} \\ 
    & = \cos{\phi} (-\rho^2\sin{\phi}\cos{\phi}\sin^2{\theta} - p^2\sin{\phi}\cos{\phi}\cos^2{\theta}) - \rho\sin{\phi} (\rho \sin^2{\phi}\cos^2{\theta + \rho \sin^2{\phi} \sin^2{\theta}}) \\ 
    & = \text{(This reduces all the way to)}  \ p^2\sin{\phi}  \ \ \text{(lol)} 
\end{align*}

anyways, putting this back into our equation would give us

$$\iiint\limits_{R}f(x,y,z) \ \mathrm{d}V = \iiint\limits_{S} f(\rho\sin{\phi}\cos{\theta}, \rho\sin{\phi}\sin{\theta}, \rho\sin{\phi}) \rho^2 \sin{\phi} \ d\rho \ d\theta \ d\phi$$

lets goo!!!!!!



\newpage
\section{Vector Fields}

\begin{center}
    In general, a vector field is a function whose domain is a set of points in $\mathbb{R}^2$ (or $\mathbb{R}^3$ in three dimensions) is a function $\mathbf{F}$ that assigns to each point $(x,y)$ in $D$ a two-dimensional vector $\mathbf{F}(x,y)$. 
\end{center}

The best way to picture a vector field is to draw the arrow represending the vector $\mathbf{F}(x,y)$ starting at the point $(x,y)$. Of course it's impossible to do this for all points $(x,y)$, but we can gain a reasonable impression of $\mathbf{F}$ by doing it for a few representative points in $D$. since $\mathbf{F}(x,y)$ is a two-dimensional vector, we can write it in terms of its \textbf{component functions} $P$ and $Q$ as follows:

$$\mathbf{F}(x,y) = P(x,y)\hat{i} + Q(x,y) \hat{j} = \langle P(x,y), Q(x,y) \rangle$$

or, for short, $\mathbf{F} = P \ \mathbf{i} + Q \  \mathbf{j}$

Notice that $P$ and $Q$ are scalar functions of two variables and are sometimes called \textbf{scalar fields} to distinguish them from vector fields.

\begin{center}
    Let $E$ be a subset of $\mathbb{R}^3$. a \textbf{vector field on} $\mathbb{R}^3$ is a function \textbf{F} that assigns each point $(x,y,z)$ in $E$ a three-dimensional vector $\mathbf{F}(x,y,z)$.
\end{center}

\subsection{Gradient Fields}

if $f$ is a scalar function of two variables, recall that $\nabla f$ is defined by $\nabla f(x,y) = f_x(x,y)\mathbf{i} + f_y(x,y)\mathbf{j} $
\\ 
\\
Therefore, $\nabla f$ is really a vector field on $\mathbb{R}^2$ and is called a \textbf{gradient vector field}. Likewise, if $f$ is a scalar function of three variables (it extends but im too lazy to type this out).

The length of the gradient vector is the value of the directional derivative of $f$ and closely spaced level curves indicate a steep graph.
\\ 
\\
A vector field $\mathbf{F}$ is called a \textbf{conservative vector field} if it is the gradient of some scalar function, that is, if there exists a function $f$ such that $\mathbf{F} = \nabla f$. in this situation $f$ is called a \textbf{potential function} for \textbf{F}.

Not all vector fields are conservative though!!


\newpage
\section{Line Integrals}

\newpage
\section{The Fundamental Theorem of Line Integrals}

\newpage
\section{Green's Theorem}

\newpage
\section{Curl and Divergence}

\newpage
\section{Parametric Surfaces}

\newpage
\section{Surface Integrals}





\end{document}