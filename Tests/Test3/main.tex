\documentclass{article} 
\usepackage[utf8]{inputenc}
\usepackage{graphicx}
\usepackage{enumitem}
\usepackage{mathtools, amssymb, amsthm} % imports amsmath

\author{Daniel Palma}
\date{\today}
\title{Test 3 Notes}

\begin{document}

\maketitle
\newpage

\tableofcontents
\newpage

\section{Change of Variables (Jacobians)}

In one-dimensional calculus we often use a change of variable to simplify an integrals.

A change of variables can also be useful in double integrals. We have already seen one example of this: conversion to polar coordinates. The new variables $r$ and $\theta$ are related to the old variables $x$ and $y$ by the equations 

$$ x = r \cos{\theta} \qquad y = r \sin{\theta} $$

and the change of variables formula can we written as 

$$\iint\limits_{R} f(x,y) \mathrm{d}A = \iint\limits_{S} f(r\cos{\theta}, r\sin{\theta})r \ dr \ d\theta$$

Where $S$ is the region in the $r\theta$-plane that corresponds to the region $R$ in the $xy$-plane

More generally, we consider a change of variables that is given by a \textbf{transformation} $T$ from the $uv$-plane to the $xy$-plane:

$$x = g(u,v) \qquad y = h(u,v)$$

or, as we sometimes write,

$$x = x(u,v) \qquad y = y(u,v)$$

We usually assume that $T$ is a \textbf{$\mathbf{C^1}$ transformation}, which means that $g$ and $h$ have continuous first-order partial derivatives.

A transformation $T$ is really just a function whose domain and range are both subsets of $\mathbb{R}^2$. If $T(u_1, v_1) = (x_1.y_1)$, then the point $(x_1,y_1)$ is called the \textbf{image} of the point $(u_1, v_1)$. If no two points have the same image, $T$ is called \textbf{one-to-one}. 

If $T$ is a one-to-one transformation, then it has an \textbf{inverse transformation $\mathbf{T^-1}$} from the $xy$-plane to the $uv$-plane and it may be possible to solve 

$$x = x(u,v) \qquad y = y(u,v)$$

for $u$ and $v$ in terms of $x$ and $y$:

$$u = G(x,y) \qquad v = H(x,y)$$

\subsection{Jacobian}

The \textbf{Jacobian} of the transformation $T$ is given by $x = g(u,v)$ and $y = h(u,v)$ is 


$$\frac{\partial(x,y)}{\partial(u,v)} = 
\begin{vmatrix}
    \frac{\partial x}{\partial u} & \frac{\partial x}{\partial v} \\[8pt]
    \frac{\partial y}{\partial u} & \frac{\partial y}{\partial v} 
\end{vmatrix} = \frac{\partial x}{\partial u} \frac{\partial y}{\partial v} - \frac{\partial x }{ \partial v} \frac{\partial y}{ \partial u}
$$


\subsection{Change of Variables in a Double Integrals}


Supopse that $T$ is a $C^1$ tranformation whose Jacobian is nonzero and that $T$ maps a region $S$ in the $uv$- plane onto a region $R$ in the $xy$-plane. Supposed that $f$ is continuous on $R$ and that $R$ and $S$ are type I or type II plane regions. Suppose also that $T$ is one-to-one, except perhaps on the boundary of $S$. Then 

$$\iint\limits_{R}f(x,y)\ \mathrm{d}A = \iint\limits_{S}f(x(u,v), y(u,v))\begin{vmatrix}
    \frac{\partial(x,y)}{\partial(u,v)}
\end{vmatrix} \ du \ dv$$

\subsubsection{Triple Integrals}

Lets use the definition of the Jacobian, extend it to three dimensions and find the formula for a triple integral and use it to derive the formula for spherical coordinates.

The Jacobian of $T$ is the following $3 \times 3 $ determinant:

$$\frac{\partial(x,y,z)}{\partial(u,v,w)} = \begin{vmatrix}
    \frac{\partial x}{\partial u} & \frac{\partial x}{\partial v} & \frac{\partial x}{ \partial w} \\[6pt]
    \frac{\partial y}{\partial u} & \frac{\partial y}{ \partial v} & \frac{\partial y}{ \partial w} \\[6pt]
    \frac{\partial z}{ \partial u} & \frac{\partial z }{\partial v} & \frac{\partial z}{ \partial w} 
\end{vmatrix}$$

this gives us the MASSIVE formula lol:

$$\iiint\limits_{R} f(x,y,z) \ \mathrm{d}V = \iiint\limits_{S} f(x(u,v,w), y(u,v,w), z(u,v,w)) \bigg\rvert \frac{\partial(x,y,z)}{\partial(u,v,w)} \bigg\rvert \ du \ dv \ dw$$

now lets use this to find the formula for triple intrgration in spherical coordinates!!!! (i'm losing my fucking mind) 

$$x =  \rho \sin{\phi} \cos{\theta} \qquad y = \rho \sin{\phi} \sin{\theta} \qquad z = \rho \cos{\phi}$$

lets compute this absolute unit of a jacobian 

\begin{align*}
    \frac{\partial(x,y,z)}{\partial(\rho,\theta,\phi)} &= \begin{vmatrix}
    \sin{\phi}\cos{\theta}  & -\rho\sin{\phi}\sin{\theta}  & \rho \cos{\phi} \cos{\theta} \\[6pt]
    \sin{\phi}\sin{\theta}   & \rho \sin{\phi} \cos{\theta} & \rho \cos{\phi}\sin{\theta}  \\[6pt]
    \cos{\phi}  & 0 & - \rho \sin(\phi)    
    \end{vmatrix} \\ 
    & = \cos{\phi} (-\rho^2\sin{\phi}\cos{\phi}\sin^2{\theta} - p^2\sin{\phi}\cos{\phi}\cos^2{\theta}) - \rho\sin{\phi} (\rho \sin^2{\phi}\cos^2{\theta + \rho \sin^2{\phi} \sin^2{\theta}}) \\ 
    & = \text{(This reduces all the way to)}  \ p^2\sin{\phi}  \ \ \text{(lol)} 
\end{align*}

anyways, putting this back into our equation would give us

$$\iiint\limits_{R}f(x,y,z) \ \mathrm{d}V = \iiint\limits_{S} f(\rho\sin{\phi}\cos{\theta}, \rho\sin{\phi}\sin{\theta}, \rho\sin{\phi}) \rho^2 \sin{\phi} \ d\rho \ d\theta \ d\phi$$

lets goo!!!!!!



\newpage
\section{Vector Fields}

\begin{center}
    In general, a vector field is a function whose domain is a set of points in $\mathbb{R}^2$ (or $\mathbb{R}^3$ in three dimensions) is a function $\mathbf{F}$ that assigns to each point $(x,y)$ in $D$ a two-dimensional vector $\mathbf{F}(x,y)$. 
\end{center}

The best way to picture a vector field is to draw the arrow represending the vector $\mathbf{F}(x,y)$ starting at the point $(x,y)$. Of course it's impossible to do this for all points $(x,y)$, but we can gain a reasonable impression of $\mathbf{F}$ by doing it for a few representative points in $D$. since $\mathbf{F}(x,y)$ is a two-dimensional vector, we can write it in terms of its \textbf{component functions} $P$ and $Q$ as follows:

$$\mathbf{F}(x,y) = P(x,y)\hat{i} + Q(x,y) \hat{j} = \langle P(x,y), Q(x,y) \rangle$$

or, for short, $\mathbf{F} = P \ \mathbf{i} + Q \  \mathbf{j}$

Notice that $P$ and $Q$ are scalar functions of two variables and are sometimes called \textbf{scalar fields} to distinguish them from vector fields.

\begin{center}
    Let $E$ be a subset of $\mathbb{R}^3$. a \textbf{vector field on} $\mathbb{R}^3$ is a function \textbf{F} that assigns each point $(x,y,z)$ in $E$ a three-dimensional vector $\mathbf{F}(x,y,z)$.
\end{center}

\subsection{Gradient Fields}

if $f$ is a scalar function of two variables, recall that $\nabla f$ is defined by $\nabla f(x,y) = f_x(x,y)\mathbf{i} + f_y(x,y)\mathbf{j} $
\\ 
\\
Therefore, $\nabla f$ is really a vector field on $\mathbb{R}^2$ and is called a \textbf{gradient vector field}. Likewise, if $f$ is a scalar function of three variables (it extends but im too lazy to type this out).

The length of the gradient vector is the value of the directional derivative of $f$ and closely spaced level curves indicate a steep graph.
\\ 
\\
A vector field $\mathbf{F}$ is called a \textbf{conservative vector field} if it is the gradient of some scalar function, that is, if there exists a function $f$ such that $\mathbf{F} = \nabla f$. in this situation $f$ is called a \textbf{potential function} for \textbf{F}.

Not all vector fields are conservative though!!


\newpage
\section{Line Integrals}

In this section we define an integral that is similar to a single integral except that instead of integrating over an interval $[a,b]$, we integrate over a curve $C$. Such integrals are called \textit{line integrals} (although curve integrals would honestly be a better term imo).

Lets start with a plane curve $C$ given by the parametric equations

$$x = x(t) \quad y = y(t) \quad a \leq t \leq b$$

or, equivalently, by the vector equation $\mathbf{r}(t) = x(t)\mathbf{i} + y(t)\mathbf{j}$, and we assume that $C$ is a smooth curve

nerd shit incoming BUT this means that $\mathbf{r'}$ is continuous and $\mathbf{r}'(t) \neq 0$.

so if we divide tha parameter interval $[a,b]$ into $n$ subintervals $[t_i-1, t_i]$ of equal width and we let $x_i = x(t_i)$ and $y_i = y(t_i)$, then the correesponding points $P_i(x_i,y_i)$ divide $C$ into $n$ subarcs with lengths $\Delta s_1, \Delta s_2, \dots, \Delta s_n$. we can do the blah blah blah riemann sum this is like the 8th time we've seen it so basically we get the limit of that and we get the following \\ 

if $f$ is defined on a smooth curve $C$ given by $x = x(t), y = y(t), a \leq t \leq b$, then the \textbf{line integral of f along c} is 

\begin{equation*}
    \int_C f(x,y) \ ds = \lim_{n \rightarrow \infty} \sum_{i=1}^{n} f(x_i^*, y_i^*) \Delta s_i
\end{equation*}

if the limit exists


but also its this

\begin{equation*}
    \int_C f(x,y) \ ds = \int^b_a f(x(t), y(t)) \ \sqrt{(\frac{dx}{dt})^2 + (\frac{dy}{dt})^2} \  dt
\end{equation*}

when we want to distinguis the original line integral $\int_C f(x,y) \ ds $ from others, we call it the \textbf{line integral with respect to arc length}.

The following formulas say that line integrals with respect to $x$ and $y$ can also be evaluated by expressing everything in terms of $t$: $x = x(t), y = y(t), dx = x'(t)dt, dy = y'(t)dt.$

\begin{equation*}
    \int_C f(x,y) \ dx = \int^b_a f(x(t), y(t)) \ x'(t) dt 
\end{equation*}

\begin{equation*}
    \int_C f(x,y) \ dy = \int^b_a f(x(t), y(t)) \ y'(t) dt
\end{equation*}

it frequently happens that line integrals with respect to x and y occur together, when this happens it's customary to abbreviate by writing 

\begin{equation*}
    \int_C P(x,y) \ dx + \int_C Q(x,y) \ dy = \int_C P(x,y) \ dx + Q(x,y) \ dy
\end{equation*}

when we are setting up a line integral, sometimes the most difficult thing is to think of a parametric representation for a curve whose geometric description is given. In particular, we often need to parameterize a line segment so its useful to remember that a vector representation of a line segment that starts at $\mathbf{r}_0$ and ends at $\mathbf{r}_1$ is given by 

\begin{equation*}
    \mathbf{r}(t) = (1- t) \mathbf{r}_0 + t\mathbf{r}_1 \qquad 0 \leq t \leq 1
\end{equation*}

\subsection{Line Integrals in Space}

We now suppose that $C$ is a smooth space curve given by the parametric equations 

\begin{equation*}
    x = x(t) \quad y = y(t) \quad z = z(t) \quad a \leq t \leq b
\end{equation*}

or by a vector equation $\mathbf{r}(t) = x(t) \mathbf{i} + y(t) \mathbf{j} + z(t) \mathbf{k}$. if $f$ is a function of three variables that is continuous on some region containing $C$, then we define the \textbf{ line integral of $\mathbf{f}$ along $\mathbf{C}$} with respect to arc length in a similar manner to that for plane curves:

\begin{equation*}
    \int_C f(x,y,z) \ ds = \lim_{x \rightarrow \infty} \sum_{i=1}^{n} f(x_i^*, y_i^*, z_i^*) \ \Delta s_i
\end{equation*}

and we then evaluate it as follows

\begin{equation*}
    \int_C f(x,y,z) \ ds = \int_{a}^{b} f(x(t), y(t), z(t)) \sqrt{(\frac{dx}{dt})^2 + (\frac{dy}{dt})^2 + (\frac{dz}{dt})^2} \ dt
\end{equation*}

although, observe that all of these formulas can all be written in a more compact vector notation 

\begin{equation*}
    \int_{a}^{b} f(\mathbf{r}(t)) \rvert \mathbf{r}'(t) \rvert \ dt 
\end{equation*}

for the special case $f(x,y,z) = 1$, we get 

\begin{equation*}
    \int_C dz = \int_{a}^{b} \rvert \mathbf{r}'(t) \rvert \ dt = L 
\end{equation*}

where $L$ is the length of the curve $C$

Line integrals along $C$ with respect to  $x, y, $ and $z$ can also be defined.
\\
Therefore, as with line integrals in the plane, we evaluate line integrals fo the form

\begin{equation*}
    \int_C P(x,y,z) \ dx + Q(x,y,z) \ dy + R(x,y,z) \ dz
\end{equation*}

by expressing everything $(x, y,z,d,dy,dz)$ in terms of the parameter $t$.

\subsection{Line Integrals of Vector Fields}

We define the \textbf{work} $W$ done by the force field \textbf{F} as the limit of the Riemann sums, namely

\begin{equation*}
    W = \int_C \mathbf{F}(x,y,z) \cdot \mathbf{T}(x,y,z) \ dx = \int_C \mathbf{F} \cdot \mathbf{T} \ ds
\end{equation*}

where \textbf{T}$(x,y,z)$ is the unit tangent vector at the point $(x,y,z)$ on $C$.

If the curve $C$ is given by the vector equation $\mathbf{r}(t) = x(t) \mathbf{i} + y(t) \mathbf{j} + z(t) \mathbf{k}$, then $\mathbf{T}(t) = \frac{\mathbf{r}'(t)}{\rvert \mathbf{r}'(t) \rvert}$, so using the equation from before we can rewrite it as 

\begin{equation*}
    W = \int_{a}^{b} \mathbf{F}(\mathbf{r}(t)) \cdot \mathbf{r}'(t) \ dt
\end{equation*}

this integral is often appreviated as $\int_C \mathbf{F} \cdot \ d\mathbf{r} $, therefore we make the following definition for the line integral of any continuous vector field.

Let \textbf{F} be a continuous vector field defined on a smooth curve $C$ given by a vector function $\mathbf{r}(t), a \leq t \leq b$. Then the \textbf{line integral of F along C} is 
\begin{equation*}
    \int_C \mathbf{F} \cdot \ d\mathbf{r} = \int_{a}^{b} \mathbf{F}(\mathbf{r}(t)) \cdot \mathbf{r}'(t) \ dt = \int_C \mathbf{F} \cdot \mathbf{T} \ ds
\end{equation*}

Notice, we can formally write $d \mathbf{r} = \mathbf{r}'(t) \ dt$

Finally, we note the connection between line integrals of vector fields and line integrals of scalar fields. Suppose the vector field \textbf{F} on $\mathbb{R}^3$ is given in component for by the equation $\mathbf{F} = P \mathbf{i} + Q \mathbf{j} + R \mathbf{k} $. we use the definition from before to compute its line integral along $C$: 

\begin{align*}
    \int_C \mathbf{F} \cdot d \mathbf{r} &= \int_{a}^{b} \mathbf{F}(\mathbf{r}(t)) \cdot \mathbf{r}'(t) dt \\
    &= \int_{a}^{b} (P \mathbf{i} + Q \mathbf{j} + R \mathbf{k}) \cdot ( x'(t) \mathbf{i} + y'(t) \mathbf{j} + z'(y) \mathbf{k}) \ dt \\
    &= \int_{a}^{b} [P(x(t), y(t), z(t))x'(t) + Q(x(t), y(t), z(t))y'(t) + R(x(t), y(t), z(t)) z'(t)] \ dt
\end{align*}

but importantly, notice this is actually the same integral from earlier, therefore we have 

\begin{equation*}
    \int_C \mathbf{F} \cdot d \mathbf{r} = \int_C P \ dx + Q \ dy + R \ dz \qquad \text{ where $\mathbf{F} = P  \ \mathbf{i} + Q \ \mathbf{j} + R \ \mathbf{k}$}
\end{equation*}

\newpage
\section{The Fundamental Theorem of Line Integrals}

If we think of the gradient vector $\nabla f$ of a function $f$ of two or three variables as a sort of derivative of $f$, then the following theorem can be regarded as a version of the Fundamental Theorem for line integrals. 
\\

Let $C$ be a smooth curve given by the vector function $\mathbf{r}(t)$, $a \leq t \leq b$. let $f$ be a differentiable function of two or three variables whose gradient vector $\nabla f$ is continuous on $C$. Then
\begin{equation*}
    \int_C \nabla f \cdot d \mathbf{r} = f(\mathbf{r}(b)) - f(\mathbf{r}(a))
\end{equation*}

this says that we can evaluate the line integral of a conservative vector field (the gradient vector field of the potential function $f$) simply by knowing the value of $f$ at the endpoints of $C$. In fact, this says that the line integral of $\nabla f$ is the net change in $f$. if $f$ is a function of two variables and $C$ is a plane curve with initial pint $A(x_1, y_1)$ and terminal point $B(x_2,y_2)$ then the formula from before becomes 

\begin{equation*}
    \int_C \nabla f \cdot d \mathbf{r} = f(x_2,y_2) - f(x_1, y_1)
\end{equation*}

if $f$ is a function of three variables and $C$ is a space curve joining the point $A(x_1, y_1, z_1)$ to the point $B(x_2, y_2, z_2)$, then we have

\begin{equation*}
    \int_C \nabla f \cdot d \mathbf{r} = f(x_2,y_z,z_2) - f(x_1, y_1, z_1)
\end{equation*}

\subsection{Independence of Path}

Suppose $C_1$ and $C_2$ are two piecewise-smooth curves (which are called \textbf{paths}) that have the same initial point $A$ and terminal point $B$. One implication of the theorem from above is that 

\begin{equation*}
    \int_{C_1} \nabla f \cdot d \mathbf{r} = \int_{C_2} \nabla f \cdot d \mathbf{r} 
\end{equation*}

so whenever $\nabla f$ is continuous, the line integral of a \textit{conservative} vector field depends only on the initial point and the terminal point of a curve.

In general, if \textbf{F} is a continuous vector field with domain $D$, we say that the line integral $\int_C \mathbf{F} \cdot d \mathbf{r}$ is \textbf{independent of path} if $\int_{C_1} \mathbf{F} \cdot d \mathbf{r} = \int_{C_2} \nabla f \cdot d \mathbf{r}$ for any two paths $C_1$ and $C_2$ in $D$ that have the same initial points and the same terminal points. With this terminology, we can say that \textit{line integrals of conservative vector fields are independent of path.}

A curve is called \textbf{closed} if its terminal point coincides with its initial point, that is, $\mathbf{r}(b) = \mathbf{r}(a)$.

\begin{center}
    $\int_C \mathbf{F} \cdot d \mathbf{r}$ is independent of path in $D$ if and only if $\int_C \mathbf{F} \cdot d \mathbf{r} = 0$ for every closed path $C$ in $D$. 
\end{center}

\begin{center}
    Suppose $\mathbf{F}$ is a vector field that is continuous on an open connected region $D$. if $\int_C \mathbf{F} \cdot d \mathbf{r}$ is independent of path in $D$, then $\mathbf{F}$ is a conservative vector field on $D$; that is, there exists is a function $f$ such that $\nabla f = \mathbf{F}$. 
\end{center}

\subsection{Conservative Vector Fields}

If $\mathbf{F}(x,y) = P(x,y) \ \mathbf{i}  + Q(x,y) \ \mathbf{j}$ is a conservative vector fiekd, where $P$ and $Q$ have continuous first-order partial derivatives on a domain $D$, then throughout $D$ we have 

\begin{equation*}
    \frac{\partial P}{\partial y} = \frac{\partial Q}{\partial x}
\end{equation*}

a \textbf{simple curve} does not intersect itself anywhere between its end points

a \textbf{simply-connected region} in the plane is a connected region $D$  such that every simple closed curve in $D$ encloses only points that are in $D$.

\begin{center}
    Let $\mathbf{F} = P \ \mathbf{i} + Q \ \mathbf{j}$ be a vector field on an open simply-connected region $D$. Suppose that $P$ and $Q$ have continuous first-order partial derivatives and 

    \begin{equation*}
        \frac{\partial P}{ \partial y} = \frac{\partial Q}{ \partial x} \qquad \text{throughout } D 
    \end{equation*}
\end{center}


\newpage
\section{Green's Theorem}



\newpage
\section{Curl and Divergence}

\newpage
\section{Parametric Surfaces}

\newpage
\section{Surface Integrals}





\end{document}