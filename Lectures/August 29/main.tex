\documentclass{article} 
\usepackage[utf8]{inputenc}
\usepackage{graphicx}
\usepackage{enumitem}
\usepackage{mathtools, amssymb, amsthm} % imports amsmath

\author{Daniel Palma}
\date{\today}
\title{Lecture Notes Calculus 3}

\begin{document}

\maketitle
\newpage

\tableofcontents
\newpage

\section{Orthongonality}

The word orthogonal is an extension of the idea of perpendicularity to things that dont have a direction.


\begin{center}
    orthongonal $\approx$ perpendicular $\approx$ normal 
\end{center}
perpendicular applies to geometric objects meanwhule normal applies to vector objects

\begin{center}
    $\vec{a}$ and $\vec{b}$ are \underline{orthongonal} if and only if $\vec{a} \cdot \vec{b} = 0$

    \bigskip

    $\vec{O}$ is orthogonal to all vectors.
\end{center}

\newpage
\section{Directional Cosines}

These are the cosines of the angles that a particular vector makes with the three different positive axes.

\bigskip

definitions:
\begin{itemize}
    \item Angle with positive x-axis: Alpha $\alpha$
    \item Angle with positive y-axis: Beta $\beta$
    \item Angle with positive z-axis: Gamma $\gamma$
\end{itemize}

\begin{center}
    given $\vec{a} = <x,y,z>$
\end{center}

$$\cos\alpha = \frac{\vec{a} \cdot \hat{i}}{|\vec{a}| |\hat{i}|} = \frac{x}{|\vec{a}|}$$

$$\cos\beta = \frac{\vec{a} \cdot \hat{j}}{|\vec{a}| |\hat{j}|} = \frac{y}{|\vec{a}|}$$

$$\cos\gamma = \frac{\vec{a} \cdot \hat{j}}{|\vec{a}| |\hat{j}|} = \frac{z}{|\vec{a}|}$$

another way to find the directional cosines is to find the unit vector, and each component will correspond to the appropriate directional cosine

\newpage
\section{Projections}

Scalar Projection of $\vec{a}$ onto $\vec{b}$ = $comp_{\vec{b}}\vec{a}$

\begin{center}
    
\end{center}

\end{document}