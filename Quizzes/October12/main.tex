\documentclass{article} 
\usepackage[utf8]{inputenc}
\usepackage{graphicx}
\newtheorem{theorem}{Theorem}
\usepackage{enumitem}
\usepackage{mathtools, amssymb, amsthm} % imports amsmath

\author{Daniel Palma}
\date{\today}
\title{Quiz 10/12/2023}

\begin{document}

\maketitle
\newpage

\tableofcontents
\newpage

\section{Local Min/Max \& Saddles}

\subsection{First Derivative Test}


If a differentiable function $f$ has a local maximum or minimum at $(a,b)$ then the following is true $(\nabla f) \bigg\rvert_{(a,b)} = \langle 0,0 \rangle$


\subsection{Second Derivative Test}

If $(a,b)$ is a critical point of f, meaning $(\nabla f) \bigg\rvert_{(a,b)} = \langle 0,0 \rangle$, then the following statements are true for the second Derivative test, $D$:

$D = f_{xx}(a,b)f_{yy}(a,b) - (f_{xy}(a,b))^2$

\begin{itemize}
    \item If$ D > 0$ and $f_{xx}(a,b) > 0$ then $f(a,b)$ is a local minimum
    \item If $D > 0$ and $f_{xx}(a,b) < 0 $ then $f(a,b)$ is a local maximum
    \item If $D < 0$ then $f(a,b)$ is a saddle point
    \item If $D = 0$ then the test is inconclusive
\end{itemize}

\section{Directional Derivatives}

The Directional Derivative of F in the direction u is denoted as follows:

$$D_uf = \nabla f \cdot \vec{u}$$

thus, at any point $(x_0, y_0)$

$$D_uf_{(x_0, y_0)} = f_x(x_0, y_0)u_x + f_y(x_0, y_0)u_y$$

a gradient / vector of three variables follows the same conventions

$$D_uf_{(x_0, y_0, z_0)} = f_x(x_0, y_0, z_0)u_x + f_y(x_0, y_0, z_0)u_y + f_z(x_0, y_0, z_0)u_z$$

\section{Gradient Vector}

the gradient vector of a differentiable function $f$ is denoted as follows:

$$\nabla f = \langle f_x, f_y \rangle$$

\subsection{Properties of the Gradient Vector}

If $\gamma$ is the angle between $\nabla f$ and $\vec{u}$ then the following is true:

$$D_uf = |\nabla f| cos(\gamma)$$

this gives us the following properties:

\begin{itemize}
    \item The function f increases the fastest when $\vec{u}$ is in the same direction as $\nabla f$, thus the maximum increase rate of f is $|\nabla f|$
    \item The function f decreases the fastest when $\vec{u}$ is in the opposite direction as $\nabla f$, thus the maximum decrease rate of f is $-|\nabla f|$
    \item Since the function $f$ does not change along level curve or surfaces, ($D_uf = 0$) then $\nabla f$ is perpendicular to the level curves or level surfaces
\end{itemize}

this means that if asked to find the direction of maximum increase at an arbitrary point $(x_0,y_0)$, the answer would be the direction of the gradient vector at that point.

$\nabla f = \langle f_x, f_y \rangle$ 

(so to find any point just plug into this)

\end{document}